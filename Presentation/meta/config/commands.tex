%===============================================================================
% Zweck: KTR-Meta-Vorlage
%===============================================================================
\usepackage[english,ngerman]{babel}

\makeatletter
\newcommand{\newlanguagecommand}[1]{%
  \newcommand#1{%
    \@ifundefined{\string#1\languagename}
      {``No def of \texttt{\string#1} for \languagename''}
      {\@nameuse{\string#1\languagename}}%
  }%
}
\newcommand{\addtolanguagecommand}[3]{%
  \@namedef{\string#1#2}{#3}}
\makeatother

\newlanguagecommand{\algo}
\addtolanguagecommand{\algo}{english}{Algorithm}
\addtolanguagecommand{\algo}{ngerman}{Algorithmus}
\newlanguagecommand{\loa}
\addtolanguagecommand{\loa}{english}{List of Algorithms}
\addtolanguagecommand{\loa}{ngerman}{Algorithmen}
\newlanguagecommand{\abbr}
\addtolanguagecommand{\abbr}{english}{List of Abbreviations}
\addtolanguagecommand{\abbr}{ngerman}{Abk\"urzungsverzeichnis}
\newlanguagecommand{\losymbols}
\addtolanguagecommand{\losymbols}{english}{List of Symbols}
\addtolanguagecommand{\losymbols}{ngerman}{Symbolverzeichnis}
\newlanguagecommand{\uni}
\addtolanguagecommand{\uni}{english}{University of Bamberg}
\addtolanguagecommand{\uni}{ngerman}{Otto-Friedrich-Universit\"at Bamberg}
\newlanguagecommand{\chair}
\addtolanguagecommand{\chair}{english}{Professorship for Computer Science}
\addtolanguagecommand{\chair}{ngerman}{Professur f\"ur Informatik}
\newlanguagecommand{\chairsub}
\addtolanguagecommand{\chairsub}{english}{Communication Services, Telecommunication \ifpresentation\else\\[.5em]\fi%
Systems and Computer Networks}
\addtolanguagecommand{\chairsub}{ngerman}{insbesondere Kommunikationsdienste,\ifpresentation\else\\[.5em]\fi%
Telekommunikationsdienste und Rechnernetze}
\newlanguagecommand{\seminar}
\addtolanguagecommand{\seminar}{english}{Seminar on}
\addtolanguagecommand{\seminar}{ngerman}{Ausarbeitung des KTR-Seminars}
\newlanguagecommand{\project}
\addtolanguagecommand{\project}{english}{Project on}
\addtolanguagecommand{\project}{ngerman}{Ausarbeitung des KTR-Projekts}
\newlanguagecommand{\topic}
\addtolanguagecommand{\topic}{english}{Topic}
\addtolanguagecommand{\topic}{ngerman}{Thema}
\newlanguagecommand{\submitter}
\addtolanguagecommand{\submitter}{english}{Submitted by}
\addtolanguagecommand{\submitter}{ngerman}{Vorgelegt von}
\newlanguagecommand{\lsupervisor}
\addtolanguagecommand{\lsupervisor}{english}{Supervisor}
\addtolanguagecommand{\lsupervisor}{ngerman}{Betreuer}

\newif\ifgit
\newif\ifseminar
\newif\ifpresentation
\newif\ifposter
\newif\ifthesis
\newif\iftodo
% Input files from meta package
\IfFileExists{config/metainfo}{%Meta info
%Necessary Information
\author[Initials]{Marcel Gro\ss{}mann, Stefan Kolb, Dr. Andreas Sch\"onberger, Gabriel Nikol}
\title{Container Virtualization}
\subtitle{Docker by Example}
%The day of the presentation
\date{\today}

%Optional Information
\subject{Docker by Example}
\keywords{docker}

\institute[]{\chair\\ \chairsub}

\titlegraphic{\includegraphics[width=13mm,height=13mm]{image/logo}}

\gittrue
\presentationtrue
}{\gitfalse}
\usepackage[utf8x]{inputenc}
\usepackage{lmodern}
\usepackage[T1]{fontenc}
\ifgit
\ifpresentation
  \usepackage[local]{gitinfo2}
  \else
  \ifthesis
  \usepackage[local]{gitinfo2}
  \else
  \usepackage[local,mark]{gitinfo2}
  \fi
\fi
\fi
\ifthesis
  \usepackage{setspace}
  \usepackage{wallpaper}
  \usepackage{appendix}
\fi
\usepackage{etex}
\usepackage{latexsym}
\usepackage{ae}
\usepackage{color}
%% Mathe und Formeln
\usepackage{amsmath}
\usepackage{calc}
\usepackage{amssymb}
\usepackage{amsthm}
\usepackage{amsfonts}
\usepackage{dsfont}
\usepackage[nice]{nicefrac}
\usepackage{cancel}  %%druchstreichen von Formeln
\usepackage{latexsym,marvosym,wasysym}
\usepackage{ucs}
\usepackage{ltxtable}
\usepackage{ragged2e}
%%   Fuer anspruchsvolle Tabellen   %%
\usepackage{longtable, colortbl}
\usepackage{multicol, multirow}
\ifposter
\usepackage{url}
\else
\providecommand\phantomsection{}
\usepackage{hyperref}
\fi
\usepackage[numbers]{natbib}
\usepackage{lscape}
\iftodo
\usepackage{todonotes}
\else
\usepackage[disable]{todonotes}
\fi
%% Graphic
\usepackage[font=footnotesize]{subfig}
\usepackage{graphicx}
\usepackage{float}
\usepackage{tikz}
\usepackage{pgfplots}
\usetikzlibrary{calc,arrows,fit,positioning,trees,intersections,backgrounds,shadows,decorations,decorations.text,decorations.markings,decorations.shapes,decorations.pathmorphing,shapes,patterns,fadings}

\pdfcompresslevel=9

% Code-Hervorhebung
% Quellcode
\usepackage{verbatim}            % Quellcode einbinden (\verbatiminput) standardpaket
\usepackage{moreverb}
% PseudoCode
\ifthesis
\usepackage[chapter]{algorithm}
\else
\usepackage{algorithm}
\fi
\usepackage{algpseudocode}
%\usepackage{algorithmicx}
\ifposter
%\usepackage[vlined]{algorithm2e}
\usepackage{times, relsize, booktabs, caption, helvet, paralist}
\fi

\floatname{algorithm}{\algo}
\algrenewcommand{\algorithmiccomment}[1]{\hskip1em\textcolor{gray!60}{$\rhd$ #1}}
\renewcommand{\listalgorithmname}{\loa}
\def\algorithmautorefname{\algo}


%%   intoc zur Aufnhame des Abkuerzungs- und Symbolverzeichnisses ins Inhaltsverzeichnis
\ifposter
\else
\usepackage[intoc]{nomencl}
\setlength{\nomlabelwidth}{.20\hsize}
%\renewcommand{\nomlabel}[1]{#1 \dotfill}
\setlength{\nomitemsep}{-\parsep}
\makenomenclature
\renewcommand{\nomname}{\abbr}
\newcommand{\nomaltname}{\losymbols}
\newcommand{\nomaltpreamble}{}
\newcommand{\nomaltpostamble}{}
\newcommand{\usetwonomenclatures}{\nomenclature[\switchnomitem]{}{}}
\newcommand{\switchnomitem}{R}
\renewcommand{\nomgroup}[1]{%
\ifthenelse{\equal{#1}{\switchnomitem}}{\switchnomalt}{}}
\newcommand{\switchnomalt}{%
\end{thenomenclature}
\newpage
\renewcommand{\nomname}{\nomaltname}
\renewcommand{\nompreamble}{\nomaltpreamble}
\renewcommand{\nompostamble}{\nomaltpostamble}
\begin{thenomenclature}
}

%\renewcommand{\nomname}{\abbr}

%%   Hervorhebung der Abkuerzungsbuchstaben   %%
\usepackage[normalem]{ulem}
\newcommand{\m}[1]{\uline{#1}}
\fi
\ifthesis
  %% Stichwortverzeichnis
  \usepackage{makeidx}
  \makeindex
\fi

\ifpresentation
\else
\ifposter
\else
% ausf\"{u}hrlichere Fehlermeldungen
\errorcontextlines=999
%
% Page-Layout: A4 aus Header
% Alternative
\setlength\headheight{14pt}
\setlength\topmargin{-15,4mm}
\setlength\oddsidemargin{-0,4mm}
\setlength\evensidemargin{-0,4mm}
\setlength\textwidth{160mm}
\setlength\textheight{252mm}
%
%% Absatzeinstellungen
\setlength\parindent{0mm}
\setlength\parskip{2ex}
\fi
\fi

\ifthesis
\usepackage[automark]{scrpage2}            % Kopf und Fusszeilen-Layout
\else
\usepackage{fancyhdr}
\fi

\usepackage{listings}
\usepackage{pifont}
\usepackage{fourier}
\usepackage{menukeys}

\definecolor{unibablueI}{HTML}{00457D}
\definecolor{unibablueII}{HTML}{336A97}
\definecolor{unibablueIII}{HTML}{6690B1}
\definecolor{unibablueIV}{HTML}{99B5CB}
\definecolor{unibablueV}{HTML}{CCDAE5}

\definecolor{unibayellowI}{HTML}{FFD300}
\definecolor{unibayellowII}{HTML}{FFDC33}
\definecolor{unibayellowIII}{HTML}{FFE566}
\definecolor{unibayellowIV}{HTML}{FFED99}
\definecolor{unibayellowV}{HTML}{FFF6CC}

\definecolor{unibagreenI}{HTML}{97BF0D}
\definecolor{unibagreenII}{HTML}{ACCC3D}
\definecolor{unibagreenIII}{HTML}{C1D86E}
\definecolor{unibagreenIV}{HTML}{D5E59E}
\definecolor{unibagreenV}{HTML}{EAF2CF}
%Not CD, darker versions
\definecolor{nounibagreenI}{HTML}{82A50B}
\definecolor{nounibagreenII}{HTML}{708C0A}

\definecolor{unibaredI}{HTML}{E6444F}
\definecolor{unibaredII}{HTML}{EB6972}
\definecolor{unibaredIII}{HTML}{F08F95}
\definecolor{unibaredIV}{HTML}{F5B4B8}
\definecolor{unibaredV}{HTML}{FADADC}
%Not CD, darker versions
\definecolor{nounibaredI}{HTML}{CC3D47}
\definecolor{nounibaredII}{HTML}{B3363E}

\definecolor{unibagrayI}{HTML}{878783}
\definecolor{unibagrayII}{HTML}{9F9F9C}
\definecolor{unibagrayIII}{HTML}{B7B7B5}
\definecolor{unibagrayIV}{HTML}{CFCFCE}
\definecolor{unibagrayV}{HTML}{E7E7E6}

\ifgit
  \newcommand{\gitkeys}{\gitAbbrevHash, \gitAuthorIsoDate, \gitAuthorName }
\fi

\ifpresentation
\usetheme{UniBa\ratio}
%\usefonttheme{
%	default | professionalfonts | serif |
%	structurebold | structureitalicserif |
%	structuresmallcapsserif
%}
\usefonttheme{professionalfonts}
%\useinnertheme{
%	circles | default | inmargin |
%	rectangles | rounded
%}
\useinnertheme{rectangles}
%\useoutertheme{
%	default | infolines | miniframes |
%	shadow | sidebar | smoothbars |
%	smoothtree | split | tree
%}
%\useoutertheme{split}
\setbeamercovered{transparent}

% Without navigation symbols
\beamertemplatenavigationsymbolsempty
\fi

\makeatletter
\ifposter
\else
\hypersetup{pdftitle={\@title}, pdfauthor={\@author}, linktoc=page, pdfborder={0 0 0 [3 3]}, breaklinks=true, linkbordercolor=unibablueI, menubordercolor=unibablueI, urlbordercolor=unibablueI, citebordercolor=unibablueI, filebordercolor=unibablueI}
\fi
%% Define a new 'leo' style for the package that will use a smaller font.
\def\url@leostyle{%
  \@ifundefined{selectfont}{\def\UrlFont{\sf}}{\def\UrlFont{\small\ttfamily}}}
\makeatother
%% Now actually use the newly defined style.
\urlstyle{leo}


\graphicspath{{images/},{\meta/config/images/},{\meta/images/}}
\pgfplotsset{compat=1.9}

\ifpresentation
\else
\makeatletter
\renewcommand{\maketitle} {
\begin{titlepage}
\ifthesis
\ThisCenterWallPaper{1}{\meta/config/images/titlepage.pdf}

\setstretch{1.2}
\vspace*{55mm}
\begin{minipage}[t]{2cm}
\textsc{Thema:}
\end{minipage}
\begin{minipage}[t]{12cm}
\textbf{\Large \@title}\\[10mm]
\textbf{\large \@subtitle \normalsize}
\end{minipage}\\[25mm]
\centering
\Huge \textbf{\degree arbeit}\\
\vspace{1cm}
\Large
im Studiengang \studycourse\ der Fakultät Wirtschaftsinformatik und Angewandte Informatik der Otto-Friedrich-Universität Bamberg\\
\normalsize
\vfill
\begin{flushleft}
\begin{tabbing}
xxxxxxxxxxxxxxx\=xxxxxxxxxxxxxx\kill
Verfasser: \> \@author\\
Themensteller:\> \advisor \\
Abgabedatum:\> \@date\\
\end{tabbing}
\end{flushleft}
\else
  \centering
    \begin{minipage}[t]{16cm}
      \hfill
      \begin{minipage}{12cm}
            \centering
        \uni
        \\[12pt]%
        {\Large \chair\\[.5em]%
        \large \chairsub}%
      \end{minipage}
      \hfill
      \begin{minipage}{3cm}
        \includegraphics[height=28mm]{\meta/config/images/logo} %height=26mm
      \end{minipage}
    \end{minipage}\\[70pt]%[50pt]
    {\Large\bf \ifseminar\seminar\else\project\fi}\\[36pt]
    {\LARGE \@title}\\[80pt]
    \ifseminar%
    {\Large\bf \topic:}\\[36pt]
    {\LARGE\bf \subtitle}\\
    \fi%
    \vfill
    \begin{minipage}{\textwidth}
      \center
      \submitter:\\
      {\Large \@author \\[18pt]}
      \lsupervisor: \supervisor \\[12pt]
      Bamberg, \@date\\
      \semester
    \end{minipage}
\fi
\end{titlepage}
}
\makeatother
\fi

\ifgit
  \renewcommand{\gitMarkFormat}{\color{unibagrayI}\ifpresentation\tiny\else\small\fi\sffamily}
\fi

\ifthesis
% Schönere Kapitel?
\renewcommand*{\chapterformat}{%
  \thechapter\enskip
  \textcolor{gray!50}{\rule[-\dp\strutbox]{2pt}{\baselineskip}}\enskip
}
\renewcommand{\headfont}{\normalfont\sffamily\itshape}    % Kolumnentitel serifenlos
\renewcommand{\pnumfont}{\normalfont\sffamily}    % Seitennummern serifenlos
\pagestyle{scrheadings}
%\pagestyle{scrplain}
\ihead[]{\headmark}              % Kolumnentitel immer oben innen
\chead[]{}                       % Mitte leer lassen
\ohead[\pagemark]{\pagemark}     % Seitennummern immer oben aussen
%\ohead[]{}
\ofoot[]{}                       % Seitennummern in der Fusszeile loeschen
\cfoot[]{\ifgit \gitMarkFormat{\gitMarkPref\,\textbullet{}\,Branch: \gitBranch\,@\,\gitAbbrevHash{} \textbullet{} Release:\gitReln{} (\gitAuthorDate)}\fi}                       % Seitennummern in der Fusszeile loeschen
\fi

\numberwithin{equation}{section}
%
%===============================================================================
% zentrale Layout-Angaben und Befehle
%===============================================================================
%
%#1 Breite
%#2 Datei (liegt im image Verzeichnis)
%#3 Beschriftung
%#4 Label fuer Referenzierung
\newcommand{\image}[4]{%
\begin{figure}[H]%
\centering%
\includegraphics[width=#1]{#2}%
\caption{#3}%
\label{#4}%
\end{figure}%
}

%#1 Breite
%#2 Datei (liegt im image Verzeichnis)
%#3 Beschriftung
%#4 Label fuer Referenzierung
\newcommand{\pic}[2]{
\begin{figure}[H]
\centering
\includegraphics[width=#1]{#2}
\end{figure}
}


%#1 Datei (liegt im graphic Verzeichnis)
%#2 Beschriftung
%#3 Label fuer Referenzierung
%#4 Skalierungsfaktor
\newcommand{\scaletikzimage}[4]{%
\begin{figure}[H]%
\centering%
\scalebox{#4}{%
\IfFileExists{graphic/#1.tikz}{\input{graphic/#1.tikz}}{
\IfFileExists{\meta/exampleGraphic/#1.tikz}{\input{\meta/exampleGraphic/#1.tikz}}{%
\colorbox{red}{Put your tikz file in the \texttt{graphic} folder}%
}}}%
\caption{#2}%
\label{#3}%
\end{figure}
}

% You must include \usepackage[font=footnotesize]{subfig} to use this command
% #1 relative width of both figures at most 0.5
% #2 picture one in /taskXX/P1
% #3 caption of figure 1
% #4 label of figure 1
% #5 picture two in /taskXX/P2
% #6 caption of figure 2
% #7 label of figure 2
% #8 overall caption
% #9 overall label
\newcommand{\twofigures}[9]{%
  \begin{figure}[H]%
    \centerline{%
      \subfloat[#3]{%
        \includegraphics[width=#1\textwidth]{#2}%
        \label{#4}%
      }%
      \hfil%
      \subfloat[#6]{%
        \includegraphics[width=#1\textwidth]{#5}%
        \label{#7}%
      }%
    }%
    \caption{#8}%
    \label{#9}%
  \end{figure}%
}

%#1 algorithm name
%#2 algorithm label
%#3 file name in code-folder
\newcommand{\pseudo}[3]{%
\small%
\begin{algorithm}[H]%
\caption{#1}%
\label{#2}%
\IfFileExists{code/#3.tex}{\input{code/#3.tex}}{%
\IfFileExists{\meta/exampleCode/#3.tex}{\input{\meta/exampleCode/#3.tex}}{%
\colorbox{red}{Put your code file in the \texttt{code} folder}%
}}%
\end{algorithm}%
\normalsize%
}

\newcounter{saveenumi}
\newcommand{\seti}{\setcounter{saveenumi}{\value{enumi}}}
\newcommand{\conti}{\setcounter{enumi}{\value{saveenumi}}}

\ifpresentation
\resetcounteronoverlays{saveenumi}
\fi

\ifthesis
\makeatletter
\newcommand{\erklaerung}{
\newpage
\section*{Eidesstattliche Erklärung}
\vspace{25mm}

Ich erkläre hiermit gemäß § 17 Abs. 2 APO, dass ich die vorstehende \degree arbeit selbständig verfasst und keine anderen als die angegebenen Quellen und Hilfsmittel benutzt habe.\\[20mm]

\begin{minipage}{0.4\textwidth}
\location , \@date \hfill \\
\textcolor{white}{M}
\end{minipage}
\begin{minipage}{0.6\textwidth}
\begin{flushright}
\begin{center}
\textcolor{white}{M}\ldots\ldots\ldots\ldots\ldots\ldots\ldots\ldots\ldots\ldots\ldots\ldots\\
\@author \vfill
\end{center}
\end{flushright}
\end{minipage}
\newpage
}
\makeatother
\fi

%===============================================================================
% Listing Styles
%===============================================================================
\lstset{basicstyle=\ttfamily,showstringspaces=false,commentstyle=\color{unibagrayI},keywordstyle=\color{unibablueI},breaklines=true}
\DeclareFixedFont{\ttb}{T1}{txtt}{bx}{n}{9} % for bold
\DeclareFixedFont{\ttm}{T1}{txtt}{m}{n}{9}  % for normal
\lstset{
language=Python,
basicstyle=\small,
otherkeywords={self},             % Add keywords here
keywordstyle=\small\bf\color{unibablueI},
emph={MyClass,__init__},          % Custom highlighting
emphstyle=\small\bf\color{nounibaredII},    % Custom highlighting style
stringstyle=\small\color{nounibagreenII},
commentstyle=\small\color{unibagrayI},          % Any extra options here
showstringspaces=false            %
}

\newcommand\YAMLcolonstyle{\color{nounibaredII}\mdseries}
\newcommand\YAMLkeystyle{\color{black}\bfseries}
\newcommand\YAMLvaluestyle{\color{nounibagreenII}\mdseries}

\makeatletter

% here is a macro expanding to the name of the language
% (handy if you decide to change it further down the road)
\newcommand\language@yaml{yaml}

\expandafter\expandafter\expandafter\lstdefinelanguage
\expandafter{\language@yaml}
{
  keywords={true,false,null,y,n},
  keywordstyle=\color{darkgray}\bfseries,
  basicstyle=\YAMLkeystyle,                                 % assuming a key comes first
  sensitive=false,
  comment=[l]{\#},
  morecomment=[s]{/*}{*/},
  commentstyle=\color{purple}\ttfamily,
  stringstyle=\YAMLvaluestyle\ttfamily,
  moredelim=[l][\color{orange}]{\&},
  moredelim=[l][\color{magenta}]{*},
  moredelim=**[il][\YAMLcolonstyle{:}\YAMLvaluestyle]{:},   % switch to value style at :
  morestring=[b]',
  morestring=[b]",
  literate =    {---}{{\ProcessThreeDashes}}3
                {>}{{\textcolor{red}\textgreater}}1
                {|}{{\textcolor{red}\textbar}}1
                {\ -\ }{{\mdseries\ -\ }}3,
}

% switch to key style at EOL
\lst@AddToHook{EveryLine}{\ifx\lst@language\language@yaml\YAMLkeystyle\fi}
\makeatother

\newcommand\ProcessThreeDashes{\llap{\color{cyan}\mdseries-{-}-}}

% Tikz grid
\makeatletter
\def\grd@save@target#1{%
\def\grd@target{#1}}
\def\grd@save@start#1{%
\def\grd@start{#1}}
\tikzset{
grid with coordinates/.style={
to path={%
\pgfextra{%
\edef\grd@@target{(\tikztotarget)}%
\tikz@scan@one@point\grd@save@target\grd@@target\relax
\edef\grd@@start{(\tikztostart)}%
\tikz@scan@one@point\grd@save@start\grd@@start\relax
\draw[minor help lines] (\tikztostart) grid (\tikztotarget);
\draw[middle help lines] (\tikztostart) grid (\tikztotarget);
\draw[major help lines] (\tikztostart) grid (\tikztotarget);
\grd@start
\pgfmathsetmacro{\grd@xa}{\the\pgf@x/1cm}
\pgfmathsetmacro{\grd@ya}{\the\pgf@y/1cm}
\grd@target
\pgfmathsetmacro{\grd@xb}{\the\pgf@x/1cm}
\pgfmathsetmacro{\grd@yb}{\the\pgf@y/1cm}
\pgfmathsetmacro{\grd@xc}{\grd@xa + \pgfkeysvalueof{/tikz/grid with coordinates/major step}}
\pgfmathsetmacro{\grd@yc}{\grd@ya + \pgfkeysvalueof{/tikz/grid with coordinates/major step}}
\foreach \x in {\grd@xa,\grd@xc,...,\grd@xb}
\node[anchor=north] at (\x,\grd@ya) {\pgfmathprintnumber{\x}};
\foreach \y in {\grd@ya,\grd@yc,...,\grd@yb}
\node[anchor=east] at (\grd@xa,\y) {\pgfmathprintnumber{\y}};
}
}
},
minor help lines/.style={
help lines, gray!20,
step=\pgfkeysvalueof{/tikz/grid with coordinates/minor step}
},
middle help lines/.style={
help lines, gray!40,
line width=\pgfkeysvalueof{/tikz/grid with coordinates/major line width},
step=\pgfkeysvalueof{/tikz/grid with coordinates/middle step}
},
major help lines/.style={
help lines, gray!80,
line width=\pgfkeysvalueof{/tikz/grid with coordinates/major line width},
step=\pgfkeysvalueof{/tikz/grid with coordinates/major step}
},
grid with coordinates/.cd,
minor step/.initial=.1,
middle step/.initial=.5,
middle line width/.initial=.5pt,
major step/.initial=1,
major line width/.initial=1pt,
}
\makeatother

\lstdefinelanguage{JavaScript}{
  keywords={break, case, catch, continue, debugger, default, delete, do, else, false, finally, for, function, if, in, instanceof, new, null, return, switch, this, throw, true, try, typeof, var, void, while, with},
  morecomment=[l]{//},
  morecomment=[s]{/*}{*/},
  morestring=[b]',
  morestring=[b]",
  ndkeywords={class, export, boolean, throw, implements, import, this},
  keywordstyle=\color{unibablueI},
  ndkeywordstyle=\color{unibagreenI},
  identifierstyle=\color{black},
  commentstyle=\color{unibagrayI}\ttfamily,
  stringstyle=\color{unibaredI}\ttfamily,
  sensitive=true
}


%% Fancy Quotes
\makeatletter
\tikzset{%
  fancy quotes/.style={
    text width=\fq@width pt,
    align=justify,
    inner sep=1em,
    anchor=north west,
    minimum width=\textwidth,
  },
  fancy quotes width/.initial={.8\textwidth},
  fancy quotes marks/.style={
    scale=8,
    text=white,
    inner sep=0pt,
  },
  fancy quotes opening/.style={
    fancy quotes marks,
  },
  fancy quotes closing/.style={
    fancy quotes marks,
  },
  fancy quotes background/.style={
    show background rectangle,
    inner frame xsep=0pt,
    background rectangle/.style={
      fill=unibagrayIV,
      rounded corners,
    },
  }
}

\newenvironment{fancyquotes}[1][]{%
\noindent
\tikzpicture[fancy quotes background]
\node[fancy quotes opening,anchor=north west] (fq@ul) at (0,0) {``};
\tikz@scan@one@point\pgfutil@firstofone(fq@ul.east)
\pgfmathsetmacro{\fq@width}{\textwidth - 2*\pgf@x}
\node[fancy quotes,#1] (fq@txt) at (fq@ul.north west) \bgroup}
{\egroup;
\node[overlay,fancy quotes closing,anchor=east] at (fq@txt.south east) {''};
\endtikzpicture}

\makeatother

\ifpresentation
\changemenucolor{gray}{bg}{named}{unibablueV}
\changemenucolor{gray}{br}{named}{unibablueI}
\changemenucolor{gray}{txt}{named}{unibablueI}
\fi

\newcommand{\cmark}{\ding{51}}%
\newcommand{\xmark}{\ding{55}}%
