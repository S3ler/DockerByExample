%===============================================================================
% Zweck:    KTR-Präsentation-Vorlage
% Erstellt: 15.04.2013
% Update:   04.07.2016
% Autor:    M.G.
%===============================================================================
\RequirePackage[hyphens]{url}



\newcommand\ratio{169}
\documentclass[10pt,aspectratio=\ratio,
%draft,
%handout,
compress
]{beamer}

\newcommand\meta{./meta}
%===============================================================================
% Zweck: KTR-Meta-Vorlage
%===============================================================================
\usepackage[english,ngerman]{babel}

\makeatletter
\newcommand{\newlanguagecommand}[1]{%
  \newcommand#1{%
    \@ifundefined{\string#1\languagename}
      {``No def of \texttt{\string#1} for \languagename''}
      {\@nameuse{\string#1\languagename}}%
  }%
}
\newcommand{\addtolanguagecommand}[3]{%
  \@namedef{\string#1#2}{#3}}
\makeatother

\newlanguagecommand{\algo}
\addtolanguagecommand{\algo}{english}{Algorithm}
\addtolanguagecommand{\algo}{ngerman}{Algorithmus}
\newlanguagecommand{\loa}
\addtolanguagecommand{\loa}{english}{List of Algorithms}
\addtolanguagecommand{\loa}{ngerman}{Algorithmen}
\newlanguagecommand{\abbr}
\addtolanguagecommand{\abbr}{english}{List of Abbreviations}
\addtolanguagecommand{\abbr}{ngerman}{Abk\"urzungsverzeichnis}
\newlanguagecommand{\losymbols}
\addtolanguagecommand{\losymbols}{english}{List of Symbols}
\addtolanguagecommand{\losymbols}{ngerman}{Symbolverzeichnis}
\newlanguagecommand{\uni}
\addtolanguagecommand{\uni}{english}{University of Bamberg}
\addtolanguagecommand{\uni}{ngerman}{Otto-Friedrich-Universit\"at Bamberg}
\newlanguagecommand{\chair}
\addtolanguagecommand{\chair}{english}{Professorship for Computer Science}
\addtolanguagecommand{\chair}{ngerman}{Professur f\"ur Informatik}
\newlanguagecommand{\chairsub}
\addtolanguagecommand{\chairsub}{english}{Communication Services, Telecommunication \ifpresentation\else\\[.5em]\fi%
Systems and Computer Networks}
\addtolanguagecommand{\chairsub}{ngerman}{insbesondere Kommunikationsdienste,\ifpresentation\else\\[.5em]\fi%
Telekommunikationsdienste und Rechnernetze}
\newlanguagecommand{\seminar}
\addtolanguagecommand{\seminar}{english}{Seminar on}
\addtolanguagecommand{\seminar}{ngerman}{Ausarbeitung des KTR-Seminars}
\newlanguagecommand{\project}
\addtolanguagecommand{\project}{english}{Project on}
\addtolanguagecommand{\project}{ngerman}{Ausarbeitung des KTR-Projekts}
\newlanguagecommand{\topic}
\addtolanguagecommand{\topic}{english}{Topic}
\addtolanguagecommand{\topic}{ngerman}{Thema}
\newlanguagecommand{\submitter}
\addtolanguagecommand{\submitter}{english}{Submitted by}
\addtolanguagecommand{\submitter}{ngerman}{Vorgelegt von}
\newlanguagecommand{\lsupervisor}
\addtolanguagecommand{\lsupervisor}{english}{Supervisor}
\addtolanguagecommand{\lsupervisor}{ngerman}{Betreuer}

\newif\ifgit
\newif\ifseminar
\newif\ifpresentation
\newif\ifposter
\newif\ifthesis
\newif\iftodo
% Input files from meta package
\IfFileExists{config/metainfo}{%Meta info
%Necessary Information
\author[Initials]{Marcel Gro\ss{}mann, Stefan Kolb, Dr. Andreas Sch\"onberger, Gabriel Nikol}
\title{Container Virtualization}
\subtitle{Docker by Example}
%The day of the presentation
\date{\today}

%Optional Information
\subject{Docker by Example}
\keywords{docker}

\institute[]{\chair\\ \chairsub}

\titlegraphic{\includegraphics[width=13mm,height=13mm]{image/logo}}

\gittrue
\presentationtrue
}{\gitfalse}
\usepackage[utf8x]{inputenc}
\usepackage{lmodern}
\usepackage[T1]{fontenc}
\ifgit
\ifpresentation
  \usepackage[local]{gitinfo2}
  \else
  \ifthesis
  \usepackage[local]{gitinfo2}
  \else
  \usepackage[local,mark]{gitinfo2}
  \fi
\fi
\fi
\ifthesis
  \usepackage{setspace}
  \usepackage{wallpaper}
  \usepackage{appendix}
\fi
\usepackage{etex}
\usepackage{latexsym}
\usepackage{ae}
\usepackage{color}
%% Mathe und Formeln
\usepackage{amsmath}
\usepackage{calc}
\usepackage{amssymb}
\usepackage{amsthm}
\usepackage{amsfonts}
\usepackage{dsfont}
\usepackage[nice]{nicefrac}
\usepackage{cancel}  %%druchstreichen von Formeln
\usepackage{latexsym,marvosym,wasysym}
\usepackage{ucs}
\usepackage{ltxtable}
\usepackage{ragged2e}
%%   Fuer anspruchsvolle Tabellen   %%
\usepackage{longtable, colortbl}
\usepackage{multicol, multirow}
\ifposter
\usepackage{url}
\else
\providecommand\phantomsection{}
\usepackage{hyperref}
\fi
\usepackage[numbers]{natbib}
\usepackage{lscape}
\iftodo
\usepackage{todonotes}
\else
\usepackage[disable]{todonotes}
\fi
%% Graphic
\usepackage[font=footnotesize]{subfig}
\usepackage{graphicx}
\usepackage{float}
\usepackage{tikz}
\usepackage{pgfplots}
\usetikzlibrary{calc,arrows,fit,positioning,trees,intersections,backgrounds,shadows,decorations,decorations.text,decorations.markings,decorations.shapes,decorations.pathmorphing,shapes,patterns,fadings}

\pdfcompresslevel=9

% Code-Hervorhebung
% Quellcode
\usepackage{verbatim}            % Quellcode einbinden (\verbatiminput) standardpaket
\usepackage{moreverb}
% PseudoCode
\ifthesis
\usepackage[chapter]{algorithm}
\else
\usepackage{algorithm}
\fi
\usepackage{algpseudocode}
%\usepackage{algorithmicx}
\ifposter
%\usepackage[vlined]{algorithm2e}
\usepackage{times, relsize, booktabs, caption, helvet, paralist}
\fi

\floatname{algorithm}{\algo}
\algrenewcommand{\algorithmiccomment}[1]{\hskip1em\textcolor{gray!60}{$\rhd$ #1}}
\renewcommand{\listalgorithmname}{\loa}
\def\algorithmautorefname{\algo}


%%   intoc zur Aufnhame des Abkuerzungs- und Symbolverzeichnisses ins Inhaltsverzeichnis
\ifposter
\else
\usepackage[intoc]{nomencl}
\setlength{\nomlabelwidth}{.20\hsize}
%\renewcommand{\nomlabel}[1]{#1 \dotfill}
\setlength{\nomitemsep}{-\parsep}
\makenomenclature
\renewcommand{\nomname}{\abbr}
\newcommand{\nomaltname}{\losymbols}
\newcommand{\nomaltpreamble}{}
\newcommand{\nomaltpostamble}{}
\newcommand{\usetwonomenclatures}{\nomenclature[\switchnomitem]{}{}}
\newcommand{\switchnomitem}{R}
\renewcommand{\nomgroup}[1]{%
\ifthenelse{\equal{#1}{\switchnomitem}}{\switchnomalt}{}}
\newcommand{\switchnomalt}{%
\end{thenomenclature}
\newpage
\renewcommand{\nomname}{\nomaltname}
\renewcommand{\nompreamble}{\nomaltpreamble}
\renewcommand{\nompostamble}{\nomaltpostamble}
\begin{thenomenclature}
}

%\renewcommand{\nomname}{\abbr}

%%   Hervorhebung der Abkuerzungsbuchstaben   %%
\usepackage[normalem]{ulem}
\newcommand{\m}[1]{\uline{#1}}
\fi
\ifthesis
  %% Stichwortverzeichnis
  \usepackage{makeidx}
  \makeindex
\fi

\ifpresentation
\else
\ifposter
\else
% ausf\"{u}hrlichere Fehlermeldungen
\errorcontextlines=999
%
% Page-Layout: A4 aus Header
% Alternative
\setlength\headheight{14pt}
\setlength\topmargin{-15,4mm}
\setlength\oddsidemargin{-0,4mm}
\setlength\evensidemargin{-0,4mm}
\setlength\textwidth{160mm}
\setlength\textheight{252mm}
%
%% Absatzeinstellungen
\setlength\parindent{0mm}
\setlength\parskip{2ex}
\fi
\fi

\ifthesis
\usepackage[automark]{scrpage2}            % Kopf und Fusszeilen-Layout
\else
\usepackage{fancyhdr}
\fi

\usepackage{listings}
\usepackage{pifont}
\usepackage{fourier}
\usepackage{menukeys}

\definecolor{unibablueI}{HTML}{00457D}
\definecolor{unibablueII}{HTML}{336A97}
\definecolor{unibablueIII}{HTML}{6690B1}
\definecolor{unibablueIV}{HTML}{99B5CB}
\definecolor{unibablueV}{HTML}{CCDAE5}

\definecolor{unibayellowI}{HTML}{FFD300}
\definecolor{unibayellowII}{HTML}{FFDC33}
\definecolor{unibayellowIII}{HTML}{FFE566}
\definecolor{unibayellowIV}{HTML}{FFED99}
\definecolor{unibayellowV}{HTML}{FFF6CC}

\definecolor{unibagreenI}{HTML}{97BF0D}
\definecolor{unibagreenII}{HTML}{ACCC3D}
\definecolor{unibagreenIII}{HTML}{C1D86E}
\definecolor{unibagreenIV}{HTML}{D5E59E}
\definecolor{unibagreenV}{HTML}{EAF2CF}
%Not CD, darker versions
\definecolor{nounibagreenI}{HTML}{82A50B}
\definecolor{nounibagreenII}{HTML}{708C0A}

\definecolor{unibaredI}{HTML}{E6444F}
\definecolor{unibaredII}{HTML}{EB6972}
\definecolor{unibaredIII}{HTML}{F08F95}
\definecolor{unibaredIV}{HTML}{F5B4B8}
\definecolor{unibaredV}{HTML}{FADADC}
%Not CD, darker versions
\definecolor{nounibaredI}{HTML}{CC3D47}
\definecolor{nounibaredII}{HTML}{B3363E}

\definecolor{unibagrayI}{HTML}{878783}
\definecolor{unibagrayII}{HTML}{9F9F9C}
\definecolor{unibagrayIII}{HTML}{B7B7B5}
\definecolor{unibagrayIV}{HTML}{CFCFCE}
\definecolor{unibagrayV}{HTML}{E7E7E6}

\ifgit
  \newcommand{\gitkeys}{\gitAbbrevHash, \gitAuthorIsoDate, \gitAuthorName }
\fi

\ifpresentation
\usetheme{UniBa\ratio}
%\usefonttheme{
%	default | professionalfonts | serif |
%	structurebold | structureitalicserif |
%	structuresmallcapsserif
%}
\usefonttheme{professionalfonts}
%\useinnertheme{
%	circles | default | inmargin |
%	rectangles | rounded
%}
\useinnertheme{rectangles}
%\useoutertheme{
%	default | infolines | miniframes |
%	shadow | sidebar | smoothbars |
%	smoothtree | split | tree
%}
%\useoutertheme{split}
\setbeamercovered{transparent}

% Without navigation symbols
\beamertemplatenavigationsymbolsempty
\fi

\makeatletter
\ifposter
\else
\hypersetup{pdftitle={\@title}, pdfauthor={\@author}, linktoc=page, pdfborder={0 0 0 [3 3]}, breaklinks=true, linkbordercolor=unibablueI, menubordercolor=unibablueI, urlbordercolor=unibablueI, citebordercolor=unibablueI, filebordercolor=unibablueI}
\fi
%% Define a new 'leo' style for the package that will use a smaller font.
\def\url@leostyle{%
  \@ifundefined{selectfont}{\def\UrlFont{\sf}}{\def\UrlFont{\small\ttfamily}}}
\makeatother
%% Now actually use the newly defined style.
\urlstyle{leo}


\graphicspath{{images/},{\meta/config/images/},{\meta/images/}}
\pgfplotsset{compat=1.9}

\ifpresentation
\else
\makeatletter
\renewcommand{\maketitle} {
\begin{titlepage}
\ifthesis
\ThisCenterWallPaper{1}{\meta/config/images/titlepage.pdf}

\setstretch{1.2}
\vspace*{55mm}
\begin{minipage}[t]{2cm}
\textsc{Thema:}
\end{minipage}
\begin{minipage}[t]{12cm}
\textbf{\Large \@title}\\[10mm]
\textbf{\large \@subtitle \normalsize}
\end{minipage}\\[25mm]
\centering
\Huge \textbf{\degree arbeit}\\
\vspace{1cm}
\Large
im Studiengang \studycourse\ der Fakultät Wirtschaftsinformatik und Angewandte Informatik der Otto-Friedrich-Universität Bamberg\\
\normalsize
\vfill
\begin{flushleft}
\begin{tabbing}
xxxxxxxxxxxxxxx\=xxxxxxxxxxxxxx\kill
Verfasser: \> \@author\\
Themensteller:\> \advisor \\
Abgabedatum:\> \@date\\
\end{tabbing}
\end{flushleft}
\else
  \centering
    \begin{minipage}[t]{16cm}
      \hfill
      \begin{minipage}{12cm}
            \centering
        \uni
        \\[12pt]%
        {\Large \chair\\[.5em]%
        \large \chairsub}%
      \end{minipage}
      \hfill
      \begin{minipage}{3cm}
        \includegraphics[height=28mm]{\meta/config/images/logo} %height=26mm
      \end{minipage}
    \end{minipage}\\[70pt]%[50pt]
    {\Large\bf \ifseminar\seminar\else\project\fi}\\[36pt]
    {\LARGE \@title}\\[80pt]
    \ifseminar%
    {\Large\bf \topic:}\\[36pt]
    {\LARGE\bf \subtitle}\\
    \fi%
    \vfill
    \begin{minipage}{\textwidth}
      \center
      \submitter:\\
      {\Large \@author \\[18pt]}
      \lsupervisor: \supervisor \\[12pt]
      Bamberg, \@date\\
      \semester
    \end{minipage}
\fi
\end{titlepage}
}
\makeatother
\fi

\ifgit
  \renewcommand{\gitMarkFormat}{\color{unibagrayI}\ifpresentation\tiny\else\small\fi\sffamily}
\fi

\ifthesis
% Schönere Kapitel?
\renewcommand*{\chapterformat}{%
  \thechapter\enskip
  \textcolor{gray!50}{\rule[-\dp\strutbox]{2pt}{\baselineskip}}\enskip
}
\renewcommand{\headfont}{\normalfont\sffamily\itshape}    % Kolumnentitel serifenlos
\renewcommand{\pnumfont}{\normalfont\sffamily}    % Seitennummern serifenlos
\pagestyle{scrheadings}
%\pagestyle{scrplain}
\ihead[]{\headmark}              % Kolumnentitel immer oben innen
\chead[]{}                       % Mitte leer lassen
\ohead[\pagemark]{\pagemark}     % Seitennummern immer oben aussen
%\ohead[]{}
\ofoot[]{}                       % Seitennummern in der Fusszeile loeschen
\cfoot[]{\ifgit \gitMarkFormat{\gitMarkPref\,\textbullet{}\,Branch: \gitBranch\,@\,\gitAbbrevHash{} \textbullet{} Release:\gitReln{} (\gitAuthorDate)}\fi}                       % Seitennummern in der Fusszeile loeschen
\fi

\numberwithin{equation}{section}
%
%===============================================================================
% zentrale Layout-Angaben und Befehle
%===============================================================================
%
%#1 Breite
%#2 Datei (liegt im image Verzeichnis)
%#3 Beschriftung
%#4 Label fuer Referenzierung
\newcommand{\image}[4]{%
\begin{figure}[H]%
\centering%
\includegraphics[width=#1]{#2}%
\caption{#3}%
\label{#4}%
\end{figure}%
}

%#1 Breite
%#2 Datei (liegt im image Verzeichnis)
%#3 Beschriftung
%#4 Label fuer Referenzierung
\newcommand{\pic}[2]{
\begin{figure}[H]
\centering
\includegraphics[width=#1]{#2}
\end{figure}
}


%#1 Datei (liegt im graphic Verzeichnis)
%#2 Beschriftung
%#3 Label fuer Referenzierung
%#4 Skalierungsfaktor
\newcommand{\scaletikzimage}[4]{%
\begin{figure}[H]%
\centering%
\scalebox{#4}{%
\IfFileExists{graphic/#1.tikz}{\input{graphic/#1.tikz}}{
\IfFileExists{\meta/exampleGraphic/#1.tikz}{\input{\meta/exampleGraphic/#1.tikz}}{%
\colorbox{red}{Put your tikz file in the \texttt{graphic} folder}%
}}}%
\caption{#2}%
\label{#3}%
\end{figure}
}

% You must include \usepackage[font=footnotesize]{subfig} to use this command
% #1 relative width of both figures at most 0.5
% #2 picture one in /taskXX/P1
% #3 caption of figure 1
% #4 label of figure 1
% #5 picture two in /taskXX/P2
% #6 caption of figure 2
% #7 label of figure 2
% #8 overall caption
% #9 overall label
\newcommand{\twofigures}[9]{%
  \begin{figure}[H]%
    \centerline{%
      \subfloat[#3]{%
        \includegraphics[width=#1\textwidth]{#2}%
        \label{#4}%
      }%
      \hfil%
      \subfloat[#6]{%
        \includegraphics[width=#1\textwidth]{#5}%
        \label{#7}%
      }%
    }%
    \caption{#8}%
    \label{#9}%
  \end{figure}%
}

%#1 algorithm name
%#2 algorithm label
%#3 file name in code-folder
\newcommand{\pseudo}[3]{%
\small%
\begin{algorithm}[H]%
\caption{#1}%
\label{#2}%
\IfFileExists{code/#3.tex}{\input{code/#3.tex}}{%
\IfFileExists{\meta/exampleCode/#3.tex}{\input{\meta/exampleCode/#3.tex}}{%
\colorbox{red}{Put your code file in the \texttt{code} folder}%
}}%
\end{algorithm}%
\normalsize%
}

\newcounter{saveenumi}
\newcommand{\seti}{\setcounter{saveenumi}{\value{enumi}}}
\newcommand{\conti}{\setcounter{enumi}{\value{saveenumi}}}

\ifpresentation
\resetcounteronoverlays{saveenumi}
\fi

\ifthesis
\makeatletter
\newcommand{\erklaerung}{
\newpage
\section*{Eidesstattliche Erklärung}
\vspace{25mm}

Ich erkläre hiermit gemäß § 17 Abs. 2 APO, dass ich die vorstehende \degree arbeit selbständig verfasst und keine anderen als die angegebenen Quellen und Hilfsmittel benutzt habe.\\[20mm]

\begin{minipage}{0.4\textwidth}
\location , \@date \hfill \\
\textcolor{white}{M}
\end{minipage}
\begin{minipage}{0.6\textwidth}
\begin{flushright}
\begin{center}
\textcolor{white}{M}\ldots\ldots\ldots\ldots\ldots\ldots\ldots\ldots\ldots\ldots\ldots\ldots\\
\@author \vfill
\end{center}
\end{flushright}
\end{minipage}
\newpage
}
\makeatother
\fi

%===============================================================================
% Listing Styles
%===============================================================================
\lstset{basicstyle=\ttfamily,showstringspaces=false,commentstyle=\color{unibagrayI},keywordstyle=\color{unibablueI},breaklines=true}
\DeclareFixedFont{\ttb}{T1}{txtt}{bx}{n}{9} % for bold
\DeclareFixedFont{\ttm}{T1}{txtt}{m}{n}{9}  % for normal
\lstset{
language=Python,
basicstyle=\small,
otherkeywords={self},             % Add keywords here
keywordstyle=\small\bf\color{unibablueI},
emph={MyClass,__init__},          % Custom highlighting
emphstyle=\small\bf\color{nounibaredII},    % Custom highlighting style
stringstyle=\small\color{nounibagreenII},
commentstyle=\small\color{unibagrayI},          % Any extra options here
showstringspaces=false            %
}

\newcommand\YAMLcolonstyle{\color{nounibaredII}\mdseries}
\newcommand\YAMLkeystyle{\color{black}\bfseries}
\newcommand\YAMLvaluestyle{\color{nounibagreenII}\mdseries}

\makeatletter

% here is a macro expanding to the name of the language
% (handy if you decide to change it further down the road)
\newcommand\language@yaml{yaml}

\expandafter\expandafter\expandafter\lstdefinelanguage
\expandafter{\language@yaml}
{
  keywords={true,false,null,y,n},
  keywordstyle=\color{darkgray}\bfseries,
  basicstyle=\YAMLkeystyle,                                 % assuming a key comes first
  sensitive=false,
  comment=[l]{\#},
  morecomment=[s]{/*}{*/},
  commentstyle=\color{purple}\ttfamily,
  stringstyle=\YAMLvaluestyle\ttfamily,
  moredelim=[l][\color{orange}]{\&},
  moredelim=[l][\color{magenta}]{*},
  moredelim=**[il][\YAMLcolonstyle{:}\YAMLvaluestyle]{:},   % switch to value style at :
  morestring=[b]',
  morestring=[b]",
  literate =    {---}{{\ProcessThreeDashes}}3
                {>}{{\textcolor{red}\textgreater}}1
                {|}{{\textcolor{red}\textbar}}1
                {\ -\ }{{\mdseries\ -\ }}3,
}

% switch to key style at EOL
\lst@AddToHook{EveryLine}{\ifx\lst@language\language@yaml\YAMLkeystyle\fi}
\makeatother

\newcommand\ProcessThreeDashes{\llap{\color{cyan}\mdseries-{-}-}}

% Tikz grid
\makeatletter
\def\grd@save@target#1{%
\def\grd@target{#1}}
\def\grd@save@start#1{%
\def\grd@start{#1}}
\tikzset{
grid with coordinates/.style={
to path={%
\pgfextra{%
\edef\grd@@target{(\tikztotarget)}%
\tikz@scan@one@point\grd@save@target\grd@@target\relax
\edef\grd@@start{(\tikztostart)}%
\tikz@scan@one@point\grd@save@start\grd@@start\relax
\draw[minor help lines] (\tikztostart) grid (\tikztotarget);
\draw[middle help lines] (\tikztostart) grid (\tikztotarget);
\draw[major help lines] (\tikztostart) grid (\tikztotarget);
\grd@start
\pgfmathsetmacro{\grd@xa}{\the\pgf@x/1cm}
\pgfmathsetmacro{\grd@ya}{\the\pgf@y/1cm}
\grd@target
\pgfmathsetmacro{\grd@xb}{\the\pgf@x/1cm}
\pgfmathsetmacro{\grd@yb}{\the\pgf@y/1cm}
\pgfmathsetmacro{\grd@xc}{\grd@xa + \pgfkeysvalueof{/tikz/grid with coordinates/major step}}
\pgfmathsetmacro{\grd@yc}{\grd@ya + \pgfkeysvalueof{/tikz/grid with coordinates/major step}}
\foreach \x in {\grd@xa,\grd@xc,...,\grd@xb}
\node[anchor=north] at (\x,\grd@ya) {\pgfmathprintnumber{\x}};
\foreach \y in {\grd@ya,\grd@yc,...,\grd@yb}
\node[anchor=east] at (\grd@xa,\y) {\pgfmathprintnumber{\y}};
}
}
},
minor help lines/.style={
help lines, gray!20,
step=\pgfkeysvalueof{/tikz/grid with coordinates/minor step}
},
middle help lines/.style={
help lines, gray!40,
line width=\pgfkeysvalueof{/tikz/grid with coordinates/major line width},
step=\pgfkeysvalueof{/tikz/grid with coordinates/middle step}
},
major help lines/.style={
help lines, gray!80,
line width=\pgfkeysvalueof{/tikz/grid with coordinates/major line width},
step=\pgfkeysvalueof{/tikz/grid with coordinates/major step}
},
grid with coordinates/.cd,
minor step/.initial=.1,
middle step/.initial=.5,
middle line width/.initial=.5pt,
major step/.initial=1,
major line width/.initial=1pt,
}
\makeatother

\lstdefinelanguage{JavaScript}{
  keywords={break, case, catch, continue, debugger, default, delete, do, else, false, finally, for, function, if, in, instanceof, new, null, return, switch, this, throw, true, try, typeof, var, void, while, with},
  morecomment=[l]{//},
  morecomment=[s]{/*}{*/},
  morestring=[b]',
  morestring=[b]",
  ndkeywords={class, export, boolean, throw, implements, import, this},
  keywordstyle=\color{unibablueI},
  ndkeywordstyle=\color{unibagreenI},
  identifierstyle=\color{black},
  commentstyle=\color{unibagrayI}\ttfamily,
  stringstyle=\color{unibaredI}\ttfamily,
  sensitive=true
}


%% Fancy Quotes
\makeatletter
\tikzset{%
  fancy quotes/.style={
    text width=\fq@width pt,
    align=justify,
    inner sep=1em,
    anchor=north west,
    minimum width=\textwidth,
  },
  fancy quotes width/.initial={.8\textwidth},
  fancy quotes marks/.style={
    scale=8,
    text=white,
    inner sep=0pt,
  },
  fancy quotes opening/.style={
    fancy quotes marks,
  },
  fancy quotes closing/.style={
    fancy quotes marks,
  },
  fancy quotes background/.style={
    show background rectangle,
    inner frame xsep=0pt,
    background rectangle/.style={
      fill=unibagrayIV,
      rounded corners,
    },
  }
}

\newenvironment{fancyquotes}[1][]{%
\noindent
\tikzpicture[fancy quotes background]
\node[fancy quotes opening,anchor=north west] (fq@ul) at (0,0) {``};
\tikz@scan@one@point\pgfutil@firstofone(fq@ul.east)
\pgfmathsetmacro{\fq@width}{\textwidth - 2*\pgf@x}
\node[fancy quotes,#1] (fq@txt) at (fq@ul.north west) \bgroup}
{\egroup;
\node[overlay,fancy quotes closing,anchor=east] at (fq@txt.south east) {''};
\endtikzpicture}

\makeatother

\ifpresentation
\changemenucolor{gray}{bg}{named}{unibablueV}
\changemenucolor{gray}{br}{named}{unibablueI}
\changemenucolor{gray}{txt}{named}{unibablueI}
\fi

\newcommand{\cmark}{\ding{51}}%
\newcommand{\xmark}{\ding{55}}%

%\usepackage[utf8x]{inputenc}
\usepackage{lmodern}
\usepackage[T1]{fontenc}
\ifgit
\ifpresentation
  \usepackage[local]{gitinfo2}
  \else
  \ifthesis
  \usepackage[local]{gitinfo2}
  \else
  \usepackage[local,mark]{gitinfo2}
  \fi
\fi
\fi
\ifthesis
  \usepackage{setspace}
  \usepackage{wallpaper}
  \usepackage{appendix}
\fi
\usepackage{etex}
\usepackage{latexsym}
\usepackage{ae}
\usepackage{color}
%% Mathe und Formeln
\usepackage{amsmath}
\usepackage{calc}
\usepackage{amssymb}
\usepackage{amsthm}
\usepackage{amsfonts}
\usepackage{dsfont}
\usepackage[nice]{nicefrac}
\usepackage{cancel}  %%druchstreichen von Formeln
\usepackage{latexsym,marvosym,wasysym}
\usepackage{ucs}
\usepackage{ltxtable}
\usepackage{ragged2e}
%%   Fuer anspruchsvolle Tabellen   %%
\usepackage{longtable, colortbl}
\usepackage{multicol, multirow}
\ifposter
\usepackage{url}
\else
\providecommand\phantomsection{}
\usepackage{hyperref}
\fi
\usepackage[numbers]{natbib}
\usepackage{lscape}
\iftodo
\usepackage{todonotes}
\else
\usepackage[disable]{todonotes}
\fi
%% Graphic
\usepackage[font=footnotesize]{subfig}
\usepackage{graphicx}
\usepackage{float}
\usepackage{tikz}
\usepackage{pgfplots}
\usetikzlibrary{calc,arrows,fit,positioning,trees,intersections,backgrounds,shadows,decorations,decorations.text,decorations.markings,decorations.shapes,decorations.pathmorphing,shapes,patterns,fadings}

\pdfcompresslevel=9

% Code-Hervorhebung
% Quellcode
\usepackage{verbatim}            % Quellcode einbinden (\verbatiminput) standardpaket
\usepackage{moreverb}
% PseudoCode
\ifthesis
\usepackage[chapter]{algorithm}
\else
\usepackage{algorithm}
\fi
\usepackage{algpseudocode}
%\usepackage{algorithmicx}
\ifposter
%\usepackage[vlined]{algorithm2e}
\usepackage{times, relsize, booktabs, caption, helvet, paralist}
\fi

\floatname{algorithm}{\algo}
\algrenewcommand{\algorithmiccomment}[1]{\hskip1em\textcolor{gray!60}{$\rhd$ #1}}
\renewcommand{\listalgorithmname}{\loa}
\def\algorithmautorefname{\algo}


%%   intoc zur Aufnhame des Abkuerzungs- und Symbolverzeichnisses ins Inhaltsverzeichnis
\ifposter
\else
\usepackage[intoc]{nomencl}
\setlength{\nomlabelwidth}{.20\hsize}
%\renewcommand{\nomlabel}[1]{#1 \dotfill}
\setlength{\nomitemsep}{-\parsep}
\makenomenclature
\renewcommand{\nomname}{\abbr}
\newcommand{\nomaltname}{\losymbols}
\newcommand{\nomaltpreamble}{}
\newcommand{\nomaltpostamble}{}
\newcommand{\usetwonomenclatures}{\nomenclature[\switchnomitem]{}{}}
\newcommand{\switchnomitem}{R}
\renewcommand{\nomgroup}[1]{%
\ifthenelse{\equal{#1}{\switchnomitem}}{\switchnomalt}{}}
\newcommand{\switchnomalt}{%
\end{thenomenclature}
\newpage
\renewcommand{\nomname}{\nomaltname}
\renewcommand{\nompreamble}{\nomaltpreamble}
\renewcommand{\nompostamble}{\nomaltpostamble}
\begin{thenomenclature}
}

%\renewcommand{\nomname}{\abbr}

%%   Hervorhebung der Abkuerzungsbuchstaben   %%
\usepackage[normalem]{ulem}
\newcommand{\m}[1]{\uline{#1}}
\fi
\ifthesis
  %% Stichwortverzeichnis
  \usepackage{makeidx}
  \makeindex
\fi

\ifpresentation
\else
\ifposter
\else
% ausf\"{u}hrlichere Fehlermeldungen
\errorcontextlines=999
%
% Page-Layout: A4 aus Header
% Alternative
\setlength\headheight{14pt}
\setlength\topmargin{-15,4mm}
\setlength\oddsidemargin{-0,4mm}
\setlength\evensidemargin{-0,4mm}
\setlength\textwidth{160mm}
\setlength\textheight{252mm}
%
%% Absatzeinstellungen
\setlength\parindent{0mm}
\setlength\parskip{2ex}
\fi
\fi

\ifthesis
\usepackage[automark]{scrpage2}            % Kopf und Fusszeilen-Layout
\else
\usepackage{fancyhdr}
\fi

\usepackage{listings}
\usepackage{pifont}
\usepackage{fourier}
\usepackage{menukeys}


\def\signed #1{{\leavevmode\unskip\nobreak\hfil\penalty50\hskip2em
  \hbox{}\nobreak\hfil(#1)%
  \parfillskip=0pt \finalhyphendemerits=0 \endgraf}}

\newsavebox\mybox
\newenvironment{aquote}[1]
  {\savebox\mybox{#1}\begin{fancyquotes}}
  {\signed{\usebox\mybox}\end{fancyquotes}}


\input{\meta/config/hyphenation}

\setbeamertemplate{caption}[numbered]
%\numberwithin{figure}{section}

\begin{document}

  %===============================================================================
  % Zum Kompilieren latexmk ausführen.
  %	Konfiguration in texmaker: Options -> Configure Texmaker -> Quick Build -> Select Latexmk + ViewPD
  %	Entsprechende Informationen in den config/metainfo verändern
  % Zur Auswahl der Sprache im folgenden Befehl
  % ngerman für deutsch eintragen, english für Englisch.
  %===============================================================================
\selectlanguage{english}
\ifnum\ratio<169
\frame{\titlepage}
\else
\frame[plain]{\titlepage}
\fi

%\AtBeginSection[]
%{
%  \frame<handout:0>
%  {
%    \frametitle{Outline}
%    \tableofcontents[currentsection,hideallsubsections]
%  }
%}

\AtBeginSubsection[]
{
  \frame<handout:0>
  {
    \frametitle{Outline}
    \tableofcontents[sectionstyle=show/hide,subsectionstyle=show/shaded/hide,subsubsectionstyle=hide]KTR
  }
}

\AtBeginSubsubsection[]
{
  \frame<handout:0>
  {
    \frametitle{Outline}
    \tableofcontents[sectionstyle=show/hide,subsectionstyle=show/shaded/hide,subsubsectionstyle=show/shaded/hide]
  }
}

\newcommand<>{\highlighton}[1]{%
  \alt#2{\structure{#1}}{{#1}}
}

\newcommand{\icon}[1]{\pgfimage[height=1em]{#1}}

\section*{}
\phantomsection
\begin{frame}{Content}
\tableofcontents
\end{frame}
%%%%%%%%%%%%%%%%%%%%%%%%%%%%%%%%%%%%%%%%%
%%%%%%%%%% Content starts here %%%%%%%%%%
%%%%%%%%%%%%%%%%%%%%%%%%%%%%%%%%%%%%%%%%%

%\section{Logo}
%\begin{frame}{Logo}
%\framesubtitle{In Blau}

%#1 Breite
%#2 Datei (liegt im image Verzeichnis)
%#3 Beschriftung
%#4 Label fuer Referenzierung

%\image{.25\textwidth}{\meta/config/images/logo.png}{Uni-Logo}{img:logo}
%\end{frame}
\begin{frame}{What you can expect (and what not)}
What you learn
\begin{itemize}
	\item you will learn the concept of docker
	\item you will learn the most used commands (not representative)
	\item you get a toolset to dockerize your applications
	\item you get some practive with the tools (use the tools in exercises)
\end{itemize}
But: Everything you will learn  is in the docker documentation, too.
\begin{itemize}
	\item you won't be a pro with docker (this needs practice)
	\item if you only dockerize your application from time to time this is mostly enough
	\item sometime you need to look into the documentation (or stackoverflow) for special cases/challenges
	\item this is not a guide to configure proxies, webserver etc. we focus on docker
\end{itemize}

\end{frame}
\section{Docker Concepts}
\begin{frame}{Docker Concepts}

What does Docker do:
\begin{itemize}
	\item ``Docker allows you to package an application with all of its dependencies into a standardized unit for software development.``
	\item ``Docker containers wrap up a piece of software in a complete filesystem that contains everything it needs to run: code, runtime, system tools, system libraries - anything you can install on a server. This guarantess that it will always run the same, regardless of the environment it is running in.``
\end{itemize}
These artifacts are packed and can be exchanged.
\begin{itemize}
\item Build, ship and run everywhere!
\end{itemize}
\end{frame}

\begin{frame}{Background: Evolution of Virtualization}
\image{.25\textwidth}{\meta/config/images/logo.png}{Uni-Logo}{img:logo}
\begin{itemize}
\item Lightweight containers enable easy horizontal scalability.
\end{itemize}
\end{frame}
\begin{frame}{Virtual Machines vs. Containers}
\image{.25\textwidth}{\meta/config/images/logo.png}{Uni-Logo}{img:logo}
\end{frame}
\begin{frame}{Container vs. Images}
\begin{itemize}
	\item An image is blueprint for a container
	\item An image contains of a read-only set of filesystem layers
	\item An instance of an image is called container
	\item You can have many running containers of the same image
	\item Major difference: container adds a writeable filesystem layer on top of the image
\end{itemize}
\image{.10\textwidth}{\meta/config/images/logo.png}{\href{https://docs.docker.com/storage/storagedriver/}{images and containers}}{img:imagesandcontainers}
\end{frame}

\begin{frame}{Docker CLI}
\begin{itemize}
	\item Docker CLI (Command Line Interface) is the docker client
\end{itemize}
\image{.50\textwidth}{graphic/docker-architecture.png}{\href{https://docs.docker.com/engine/docker-overview}{Docker overview}}{img:imagesandcontainers}
\begin{itemize}
	\item \textbf{Daemon}: daemon of the server process to manage containers
	\item \textbf{Client}: user client to (remotely) controll the daemon
	\item \textbf{Registry}: platform for sharing and managing images
\end{itemize}
\end{frame}

\begin{frame}{Docker List Commands}

Shows \textbf{state} of containers, volumes and images. \\

Show \textbf{running} container only:
\begin{itemize}
		\item \emph{docker container ls}
\end{itemize}

Show \textbf{all} containers (both commands do the same):
\begin{itemize}
	\item  \emph{docker container ls -a}
	\item  \emph{docker container ls --all}
\end{itemize}

The same for volumes, networks and images:
\begin{itemize}
	\item docker volume ls --all
    \item docker image ls --all
    \item docker network ls
\end{itemize}
%\image{.10\textwidth}{\meta/config/images/logo.png}{example of \emph{docker container ls}}{img:dockercontainerlsexample}
\end{frame}

\begin{frame}{Docker Pull Command}
The \emph{\textbf{docker pull IMAGE}} command downloads an image from docker registry.

In our case (default) from the \textbf{public} registry - \href{https://hub.docker.com/}{docker hub}

Pull the \textbf{latest} image of busybox (default):
\begin{itemize}
	\item docker pull busybox
\end{itemize}

Pull busybox with tag \textbf{1.29.2-glibc}:
\begin{itemize}
	\item docker pull busybox:1.29.2-glibc
\end{itemize}
%Download non-locale images from Docker-Hub
%Question: How to check if the images is in downloaded?
\end{frame}


\begin{frame}{Docker Run Command}
The \href{https://docs.docker.com/engine/reference/run/}{\emph{\textbf{docker run [OPTIONS] IMAGE[:TAG]}}} command starts container from images.
\begin{itemize}
	\item docker run busybox
\end{itemize}
Start container for interactive processes (\emph{-i}) and allocate a tty (\emph{-t}). Together written as \emph{-it}.

Open busybox with shell:
\begin{itemize}
	\item docker run -it busybox
\end{itemize}

%\image{.10\textwidth}{\meta/config/images/logo.png}{\emph{docker run -it busybox} opens shell in a busybox}{img:dockerrunitbusybox}

\end{frame}

\begin{frame}{Docker Run Command contd.}
Other very important docker run \textbf{[OPTIONS]}:
Give container names (unique):
\begin{itemize}
	\item docker run -it --name busy busybox
\end{itemize}
delete container foobar after running:
\begin{itemize}
	\item docker run --rm busybox
\end{itemize}
%\image{.10\textwidth}{\meta/config/images/logo.png}{\emph{docker image inspect ubuntu}}{img:dockerimageinspectubuntu}
\end{frame}

\begin{frame}{Docker Inspect Command}
You can also inspect container, volumes, networks and images giving you detailed information:
\begin{itemize}
	\item docker container inspect ID|NAME
	\item docker volume inspect VOLUMENAME
	\item docker image inspect IMAGE
	\item docker network inspect ID|NAME
\end{itemize}
%\image{.10\textwidth}{\meta/config/images/logo.png}{\emph{docker image inspect ubuntu}}{img:dockerimageinspectubuntu}

\end{frame}

\begin{frame}{Exercise 1}
\begin{enumerate}
	\item Open a first (power)shell.
	\item download the image: \structure{\href{https://hub.docker.com/_/busybox/}{busybox}} and \structure{\href{https://hub.docker.com/_/ubuntu/}{ubuntu}}.
	
	What do you see? What is different between the two images?
	
	\item start a \textbf{ubuntu} container with the name \textbf{foo}.
	
	List all directories (\emph{ls}), create a new directory (\emph{mkdir DIRNAME}), List all directories again.
	
	\item Open a second (power)shell.
	\item List all containers.  What is the Id and name of your container?
	\item Go back to the first (power)shell and exit the container.
	
	Grateful with \emph{exit}, abort with \emph{STRG+C}.
	
	\item List \textbf{all} containers again. What changes?
	
	\item inspect the ubuntu container and image
	
	\item delete the container by using: \emph{docker container rm ID|NAME}	and list all containers again
	
\end{enumerate}
\end{frame}

\begin{frame}{Docker handling detached containers}
Normally container run in the background without interaction.


Run container without active tty with \emph{--detach} or \emph{-d} option:
\begin{itemize}
	\item docker run --detach --name influx --rm  influxdb
\end{itemize}

Inspect stdout with \emph{logs}
\begin{itemize}
	\item docker logs influx
\end{itemize}

Stop running container with
\begin{itemize}
	\item docker stop influx
\end{itemize}

Stop running container with
\begin{itemize}
	\item docker stop influx
\end{itemize}
\end{frame}


\begin{frame}{environment variables}
You can define environment variables passed to a container.

Syntax: \emph{-e=VARIABLE} or \emph{--env=VARIABLE}

Example: print out value of environment variable foo on console
\begin{itemize}
	\item docker run -it -e foo=bar ubuntu
	\item / \# echo \$foo
\end{itemize}
\image{.10\textwidth}{\meta/config/images/logo.png}{\emph{docker run -it -e foo=bar ubuntu}}{img:dockerrunitefoobarubuntu}
\end{frame}


\begin{frame}{publish ports}
By default no ports are accessible outside of container networks.
You need to make the explicitly available to the host.

\vspace{1cm}
Syntax: \textbf{-p EXTERN:INTERN} or \textbf{- -publish EXTERN:INTERN}

\vspace{1cm}
Example: start a http server on port 80
\begin{itemize}
	\item docker run -p 80:80 httpd
	\item docker run -publish 80:80 httpd
\end{itemize}

\end{frame}

\begin{frame}{Docker volume option}
Mapping of local file system into container file system.

Syntax: 
\begin{itemize}
	\item \textbf{-v LOCALFS:CONTAINERFS}
	\item \textbf{- -volume LOCALFS:CONTAINERFS}
\end{itemize}
Local file system path: C:/User/Name/html
Example: start php container with C:/User/Name/html mapped into /var/www/html
\begin{itemize}
	\item docker run -v C:/User/Name/html:/var/www/html php:apache
\end{itemize}
\end{frame}


%\begin{frame}{link}
%\end{frame}

\begin{frame}{Exercise 2}
\begin{enumerate}
	\item GoTo: \url{https://hub.docker.com/_/mysql/} and find the setable environment variables.
	\item Run all container in detached mode
	\item Run a mysql container named database with root password \textbf{foobar}.
	\item Check the logs of the mysql
	\item Stop the mysql container
	\item Run a php:7.2-apache container named webserver, publish port 80 and map directory Exercise2/html into /var/www/html
	\item Ensure that the webservice's html page is available
	\item list all running container and compare the output with Figure \autoref{img:exercise2dockercontainerls}. It should be equal.
	\image{.10\textwidth}{\meta/config/images/logo.png}{\emph{Exercise2: docker container ls}}{img:exercise2dockercontainerls}
\end{enumerate}
\end{frame}


\begin{frame}{Docker volumes}
Volumes without mounted host directory, so no absolute path needed. Increases portability of applications.


Create a volume named database.
\begin{itemize}
	\item docker volume create database
\end{itemize}
Remember: list volumes
\begin{itemize}
	\item docker volume ls
\end{itemize}

Map a volume like a file system volume by its name.
\begin{itemize}
	\item docker run -v database:/var/lib/influxdb --name influx --rm influxdb
\end{itemize}
\end{frame}

\begin{frame}{Exercise 3}
\begin{enumerate}
	\item open a first (power)shell
	\item create a volume named shared\_volume
	\item run a ubuntu image and map shared\_volume to /shared\_volume
	\item open a second (power)shell
	\item run a influxdb image and map shared\_volume to /var/lib/influxdb
	\item change back to the first (power)shell
	\item list (ls) the content of /shared\_volume
	
\end{enumerate}
\end{frame}

%\begin{frame}{Docker --link command (deprecated)}
%Linking by name like
%\end{frame}

\begin{frame}{Docker network}
Until now, container were not interconnection (except over publish ports).
We can link container via their names, this is provided by an internal docker DNS server.
For this they must be in the same docker network.

Create a network named mynet.
\begin{itemize}
	\item docker network create mynet
\end{itemize}
Remember: list networks
\begin{itemize}
	\item docker networks ls
\end{itemize}

Join a container to a network:
\begin{itemize}
	\item docker run --name somecontainer --net mynet busybox
\end{itemize}

\end{frame}

\begin{frame}{Exercise 4}
\begin{enumerate}
	\item open a first (power)shell
	\item create a network named foobar
	\item run a busybox image named foo and join the foobar
	\item open a second (power)shell
	\item run a busybox image named bar and join the foobar
	\item ping foo from the second and bar from the first (power)shell
	\item stop container foo, what happens in the bar container?
	\item inspect the foobar network
	\item stop all containers, delete the network and clean up the environment
	
\end{enumerate}
\end{frame}

\begin{frame}{Outlook}

\href{https://docs.docker.com/engine/reference/run/}{Other usefull commands:}

Execute something inside a running container - in most cases a shell:
Example: detached httpd webserver and exec into bash
\begin{itemize}
	\item docker run -d -p 80:80 --name webserver httpd
	\item sudo docker exec -i -t webserver /bin/bash
\end{itemize}


Override default entrypoint with docker run:
Example: instead of executing apache go to bash
\begin{itemize}
	\item docker run -it --entrypoint "/bin/bash" php
\end{itemize}

\end{frame}

%##################################

\section{Docker files}
%\begin{frame}{Concept of Dockerfiles}
%\begin{itemize}
%\item until now we used existing images
%\item Not really but a good concept: blueprints for images
%\item a list of commands to create a state 
%\item like setting up a server
%\end{itemize}
%%TODO add example dockerfile
%\end{frame}

\begin{frame}{Dockerfile - Motivation}
Motivation
\begin{itemize}
	\item until now we used existing images
	\item you can commit changes to created conainers, this creates new images (not addressed in this tutorial)
	\item but cumbersome to share local created images, hard to reproduce
	\item solution: declarative way to create images => Dockerfiles
\end{itemize}

\end{frame}

\begin{frame}{Dockerfile - Concept}
Concept
\begin{itemize}
	\item blueprints for images
	\item a list of commands to create a state of an image
	\item makes image creation reproducible
	\item each RUN instruction adds a delta-layer
	%https://docs.docker.com/engine/reference/builder/#format
	%TODO add example dockerfile
\end{itemize}
Instructions
\begin{itemize}
	\item single line statements and contain a keyword
	\item not case sensitive
\end{itemize}
Example:
	\image{.10\textwidth}{\meta/config/images/logo.png}{\emph{Minimalistic Dockerfile}}{img:minimalisticdockerfile}
\end{frame}

\begin{frame}{Dockerfile Instructions 1/6}
Every Dockerfile starts with an image it is based on.
For this the FROM Keyword is used.
\begin{itemize}
	\item FROM <image>[:<tag>] [AS <name>]
\end{itemize}
Example: Ubuntu is a good start for every new docker application
\begin{itemize}
	\item FROM ubuntu:latest
\end{itemize}


You can add comments to you Dockerfile with the \# (hash)
Example: A comment
\begin{itemize}
	\item \# this is comment in a Dockerfile
\end{itemize}
\end{frame}

\begin{frame}{Dockerfile Instructions 2/6}

A Dockerfile is a list of instructions to create a image.
To execute commands on the shell use the RUN keyword.
There exist two forms (shell and exec form):
shell form:
\begin{itemize}
	\item RUN <command>
\end{itemize}
exec form:
\begin{itemize}
	\item RUN ["executable", "param1", "param2"]
\end{itemize}

In the shell form you can use a \textbackslash  (backslash) to continue a single RUN instruction onto the next line. This increases readability!


Example: install git inside an ubuntu image and clean up apt list for thin layers
\begin{itemize}
	\item RUN apt-get update \&\& apt-get install -y \textbackslash \\
	git \textbackslash \\
	\&\& rm -rf /var/lib/apt/lists/*
\end{itemize}
\end{frame}


\begin{frame}{Dockerfile Instructions 3/6}
Until now you never explicitly said what application to run when you started a container.
When writing dockerfile you need to define this by using the CMD or ENTRYPOINT command. This sets the image’s main command.

CMD
\begin{itemize}
	\item CMD [“executable”, “param1”, “param2”…]
	\item CMD executable param1
\end{itemize}


ENTRYPOINT (exec form preferred)
\begin{itemize}
	\item ENTRYPOINT [“executable”, “param1”, “param2”…]
	\item ENTRYPOINT executable param1
\end{itemize}
\end{frame}

\begin{frame}{Docker CLI - Building Images}
You can build you images from a Dockerfile by using the Docker CLI. \\
General usage: docker build [OPTIONS] PATH | URL | - \\

\begin{enumerate}
\item path to Dockerfile (Default is 'PATH/Dockerfile')
Syntax:
\begin{itemize}
	\item -f path/to/Dockerfile
	\item --file path/to/Dockerfile
\end{itemize}
\item name and tag of image
Syntax:
\begin{itemize}
	\item -t name:tag
	\item --tag name:tag
\end{itemize}
\item last option is the path to the build context (mostly . )
\end{enumerate}

Example: Build a Dockerfile with buildcontext in current (power)shell directory
\begin{itemize}
	\item docker build -t foo:bar -f path/to/Dockerfile .
\end{itemize}
\end{frame}

\begin{frame}{Dockerfile - Exercise 1}
\begin{enumerate}
	\item create a new Dockerfile in the Dockerfile/Exercise1 folder
	\item use ubuntu as base image
	\item install cowsay and lolcat in one layer (optional: clean up)
	\item create an ENTRYPOINT in exec format executing \emph{/usr/games/cowsay} and \emph{"Hello"} as first parameter
	\item create a CMD in exec format with \emph{"World"} (will be second parameter to cowsay)
	\item build your image named cowsay
	\item run the image cowsay without parameter and with parameter (do not change the Dockerfile)
	\item comment out the ENTRYPOINT and CMD
	\item append a new CMD with shell format executing the following: \\ \emph{/usr/games/cowsay "What does the cow say?" | /usr/games/lolcat}
	\item build your image named lolsay
	\item run the image lolsay
\end{enumerate}
\end{frame}

\begin{frame}{Dockerfile Instructions 4/6}
\begin{itemize}
	\item ADD copy-on-write, adds layer
	\item COPY

\end{itemize}
\end{frame}

\begin{frame}{Dockerfile Instructions 5/6}
\begin{itemize}

	\item EXPOSE docker run -p 80:80/tcp -p 80:80/udp ...
\end{itemize}
\end{frame}

\begin{frame}{Dockerfile - Exercise 2}
\begin{itemize}
	\item use the directory dockerfile/exercise2
	\item user alpine:latest as base image 
	\item new dockerfile
	\item user alpine as base image yourself as maintainer
	\item install apache2 (and cleanup)
	\item copy example content to /var/www/html
	\item expose ports
	\item set entrypoint to start apache2
	\item create image named ouruser/alpine:apache2
	\item run your created image, do not forget to publish port, optional run it in detached mode
	\item check if your webside is locally reachable by your published port
\end{itemize}
\image{.10\textwidth}{\meta/config/images/logo.png}{\emph{r1234}}{img:1234}
\end{frame}

\begin{frame}{Dockerfile Outlook}
There are many more Dockerfile Instruction e.g.
\begin{itemize}
	\item WORKDIR (cd)
	\item ENV (cd)
	\item ARGS
	\item VOLUME creates a volume at runtime
	\item many more. e.g.
\end{itemize}
% https://docs.docker.com/engine/reference/builder/#notes-about-specifying-volumes
\end{frame}

\begin{frame}{Docker Compose - Motivation}
\begin{itemize}
	\item Tpyically, an application is not one service, but orchestration of multiple smaller isolated service units
\end{itemize}
\image{.10\textwidth}{\meta/config/images/logo.png}{\emph{phpmysqlapplicationoverview}}{img:phpmysqlapplicationoverview}
\end{frame}
\begin{frame}{Docker Compose - Motivation}
Multi-container apps are a hassle
\begin{itemize}
	\item Until now:
	\item pull images, build dockerfiles
	\item create networks
	\item create container with long configuration lines (volumes, publish, networks)
	\item naming is global
	\item container logs for each container by name or id
	\item for a lot of containers!!
\end{itemize}
Multi-container apps are a hassle
\begin{itemize}
	\item docker-compose up does everything automatically
	\item (after configuration of course)
\end{itemize}
\end{frame}

\begin{frame}{Docker Compose - Overview}

\begin{itemize}
	\item bla
	\item single host!!! (else kubernetes)
	\item minimal example
\end{itemize}
\end{frame}

\begin{frame}{Docker Compose CLI - Commands}

\begin{itemize}
	\item bla
	\item minimal example
\end{itemize}
\end{frame}

\begin{frame}{Docker Compose File - Aufbau}

\begin{itemize}
	\item must be named docker-compose.yml or docker-compose.yaml
	\item container become services
	\item yaml file (DexternalONT USER TAB, ONLY SPACES)
\end{itemize}
\end{frame}

\begin{frame}{Docker Compose File - Command Translation}

\begin{itemize}
	\item publish -> ports:
	\item volumes with local FS
	\item environment variable
\end{itemize}
\end{frame}

\begin{frame}{Docker Compose File - Build Images with Docker Compose}

\begin{itemize}
	\item build
	\item path to Dockerfile
	\item naming
\end{itemize}
\end{frame}



\begin{frame}{Exercise Docker Compose - Beforehand}
Let's clean up container, images, networks etc.
Execute: docker system prune etc.
\end{frame}

\begin{frame}{Exercise Docker Compose - PHP Application 1/2}


First Service: PHP with Apache2
\begin{itemize}
	\item Dockerfile
	\begin{itemize}
			\item expose port 8080
			\item create dockerfile from php:apache2
			\item copy scripts to /var/www/html
	\end{itemize}
	\item publish port 8080
	\item create a new service named phpapache2
	\item depend on mysql service
\end{itemize}
Second Service: Database with MySQL
\begin{itemize}
	\item create a new service named db\_mysql
	\item map volume
	\item define environment variables user passwd
\end{itemize}

\end{frame}

\begin{frame}{Docker Compose File - Command Translation}
networks
volumes
by default container are in the bridge network
\end{frame}

\begin{frame}{Exercise Docker Compose - PHP Application with NGINX 2/2}
User your artifact from the last exercise
% TODO provide NGINX with self-signed CA
First Service: NGINX
\begin{itemize}
	\item create a new service named nginx\_reverse\_proxy
	\item expose port 80
	\item volume conf.d/default.conf into
	\item depend on php\_apache2
	\item join an appropriate network
\end{itemize}
Edit Service: PHP with Apache2
\begin{itemize}
	\item remove the publish
\end{itemize}
Second Service: Database with MySQL
\begin{itemize}
	\item create a new service named db\_mysql
	\item map volume etc
\end{itemize}
Networks
\begin{itemize}
	\item the Database service shall only be reachable by the php\_apache2 service, so put them in the same network
	\item the php\_apache2 shall be rechable from the nginx reverse proxy but not from the outside
	\item nginx shall be accesible over port 80 from the outside
\end{itemize}
\end{frame}

%\section{Git Version}
%\begin{frame}{Git Version}\framesubtitle{\gitAbbrevHash}
%\begin{itemize}
%\item \#: \gitAbbrevHash
%\item @: \gitAuthorIsoDate
%\item \gitReferences
%\end{itemize}
%\textbf{Setup}\\
%Follow this link to %\href{https://www.ctan.org/tex-archive/macros/latex/contrib/gitinfo2}{Gitinfo 2}\\
%\textbf{git hooks}\\
%  To fill watermark at buttom, deploy gitinfo2-hook.txt to githooks: (copy and make executable) or use \texttt{make git}
%  \begin{itemize}
%      \item .git/hooks/post-checkout
%      \item .git/hooks/post-commit
%      \item .git/hooks/post-merge
%  \end{itemize}
%  \textbf{Remove watermark}\\
%  To disable watermark, remove option \texttt{[mark]} from \textbackslash usepackage[mark]\{gitinfo2\} in \textit{config/commands.tex}.
%\end{frame}


%%%%%%%%%%%%%%%%%%%%%%%%%%%%%%%%%%%%%%%%%
%%%%%%%%%% References          %%%%%%%%%%
%%%%%%%%%%%%%%%%%%%%%%%%%%%%%%%%%%%%%%%%%
\section*{}
\begin{frame}[allowframebreaks]{References}
 \def\newblock{\hskip .11em plus .33em minus .07em}
 \scriptsize
 \bibliographystyle{IEEEtran}
 %\bibliography{\meta/exampleLiterature/bib}
 \normalsize
\end{frame}

%% Last frame
\frame{
  \vspace{2cm}
  {\huge Questions ?}

  \vspace{20mm}
  \nocite*

  \begin{flushright}
    made by Gabriel Nikol

    \structure{\footnotesize{\href{https://github.com/S3ler/DockerByExample}{https://github.com/S3ler/DockerByExample}}}
  \end{flushright}
}


\end{document}
